\documentclass[10pt,twocolumn,letterpaper]{article}

\usepackage{cvpr}
\usepackage{times}
\usepackage{epsfig}
\usepackage{graphicx}
\usepackage{amsmath}
\usepackage{amssymb}

% Include other packages here, before hyperref.

% If you comment hyperref and then uncomment it, you should delete
% egpaper.aux before re-running latex.  (Or just hit 'q' on the first latex
% run, let it finish, and you should be clear).
\usepackage[pagebackref=true,breaklinks=true,letterpaper=true,colorlinks,bookmarks=false]{hyperref}

\cvprfinalcopy % *** Uncomment this line for the final submission

\def\cvprPaperID{****} % *** Enter the CVPR Paper ID here
\def\httilde{\mbox{\tt\raisebox{-.5ex}{\symbol{126}}}}

% Pages are numbered in submission mode, and unnumbered in camera-ready
\ifcvprfinal\pagestyle{empty}\fi
\begin{document}

%%%%%%%%% TITLE
\title{Disease Type Prediction(hackerearth deep learning challenge \#2)}

\author{Yumin Sun\\
{\tt\small suny@in.tum.de}
\and
Zhen Yan\\
{\tt\small zhen.yan@tum.de}
\and
Yuchang Zhang\\
{\tt\small ga63jof@mytum.de }
\and
Xiaojing Li\\
{\tt\small fourth@i1.org}
%\and
%Team Member 5\\
%{\tt\small fifth@i1.org}
}


\maketitle
%\thispagestyle{empty}

%
% Proposal I
%
\section*{Project Proposal}
We are doing the challenge presented \href{https://www.hackerearth.com/challenge/competitive/deep-learning-challenge-2/machine-learning/yes-a-question/}{here}.

\section{Introduction}
	Disease type diagnosis from X-rays is of low-cost and simple. However, lack of experienced doctors and high miss misdiagnosed rates makes it a challenge. We are trying to solve this problem using deep learning as well as classical machine learning techniques, with over 10,000 labeled data.
	
    \subsection{Related Works}
        \begin{itemize}
            \item Learning to Read Chest X-Ray Images from 16000+ Examples Using CNN \cite{dong2017learning}
        \end{itemize}

\section{Dataset}
We are using the dataset provided by hackerearth. The training data is split into two parts. One with X-ray pictures and disease labels. This other one includes general information of the patients, i.e., gender and age.

\section{Methodology}
We are planning to try different pre-trained models combined with out own self-defined layers. Apart from that, we are also planning to use general patient information as additional inputs 

\section{Outcome}
    Try to achieve over 75\% accuracy on test data, try to reach a score over 0.5 (Currently the best core is about 0.38).

{\small
\bibliographystyle{ieee}
\bibliography{bib}
}

\end{document}