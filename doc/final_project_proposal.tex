\documentclass[10pt,twocolumn,letterpaper]{article}

\usepackage{cvpr}
\usepackage{times}
\usepackage{epsfig}
\usepackage{graphicx}
\usepackage{amsmath}
\usepackage{amssymb}

% Include other packages here, before hyperref.

% If you comment hyperref and then uncomment it, you should delete
% egpaper.aux before re-running latex.  (Or just hit 'q' on the first latex
% run, let it finish, and you should be clear).
\usepackage[pagebackref=true,breaklinks=true,letterpaper=true,colorlinks,bookmarks=false]{hyperref}

\cvprfinalcopy % *** Uncomment this line for the final submission

\def\cvprPaperID{****} % *** Enter the CVPR Paper ID here
\def\httilde{\mbox{\tt\raisebox{-.5ex}{\symbol{126}}}}

% Pages are numbered in submission mode, and unnumbered in camera-ready
\ifcvprfinal\pagestyle{empty}\fi
\begin{document}

%%%%%%%%% TITLE
\title{Disease Type Prediction}

\author{Team Member 1\\
{\tt\small first@i1.org}
\and
Team Member 2\\
{\tt\small second@i1.org}
\and
Team Member 3\\
{\tt\small third@i1.org}
\and
Team Member 4\\
{\tt\small fourth@i1.org}
%\and
%Team Member 5\\
%{\tt\small fifth@i1.org}
}


\maketitle
%\thispagestyle{empty}

%
% Proposal I
%
\section*{Project Proposal}
The following bullet points and remarks are meant to guide through the process of writing this proposal. Nevertheless we expect a coherent text.

\section{Introduction}
    Explain your general idea and state the problem you are trying to solve.

    \subsection{Related Works}
        \begin{itemize}
            \item Related and previous work on your topic
            \item A small overview of the SOTA (state-of-the-art)
            \item What is new/different in your approach?
            \item $\dots$
        \end{itemize}

\section{Dataset}
    \begin{itemize}
        \item Are you working with an existing dataset or is data collection part of your project?
        \item Explain the general nature of your data (show examples if beneficial) and provide information how you collected your dataset
        \item Does your data provide the labels necessary for training?
        \item Explain possible problems with the dataset
        \item What are your inputs and outputs?
        \item $\dots$
    \end{itemize}

\section{Methodology}
    Describe all the different phases/components/steps on how you are planning to solve your previously stated problem. Try to give us a clear picture of your methodology and remember that the project's focus is on deep learning. Try to elaborate on the following points:
    \begin{itemize}
        \item Network architecture(s)
        \item Transfer learning and training from scratch
        \item Resource management (please consider GPU memory)
        \item $\dots$
    \end{itemize}

\section{Outcome}
    What is your desired outcome or aspiration for the project?

{\small
\bibliographystyle{ieee}
\bibliography{bib}
}

\end{document}